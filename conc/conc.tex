\chapter{Conclusions \& Future Work}
\label{chap:conc}

\section{Conclusions}

It is an exciting time in gravitational-wave astrophysics and numerical relativity. With the first detection of gravitational waves from a a binary neutron star and the coincident detection of electromagnetic counterparts across the EM spectrum, a lot can be discovered about these extreme systems. With many more detections on the horizon, we will need to be prepared on the theoretical side. To help with the ongoing theoretical effort of properly modelling binary compact object systems and their emissions, this thesis has attempted to make progress along two different directions.

Firstly, we have sought to improve the numerical techniques used to solve the partial differential equations that arise in binary neutron star simulations. To this effect, Chapter 2 presented a completely new scheme based on discontinuous Galerkin methods, and we tested it to solve for constant density star initial data, which contains phase transitions similar to that in neutron stars and we also tested it on multi-black hole puncture initial data, which contains multiple non-smooth points. All tests showed promising results. Currently, work is being done to extend this code by Nils Fischer and Prof. Harald Pfeiffer at the Albert Einstein Institute (AEI) in Potsdam, Germany by porting it into the task based parallel code SpECTRE (\cite{kidder2016spectre}), which is currently being developed at Cornell, Caltech and AEI to solve hyperbolic problems in numerical relativity on future exascale supercomputers using a task-based parallelism framework. The final goal will be to run our implementation on binary neutron star initial data with realistic microphysics and solve down to accuracies previously unobtainable.

Secondly, we wanted to increase the knowledge-base surrounding emissions from BNS mergers and their connection to binary parameters. Chapter 3 presented 12 state-of-art general relativistic radiation-hydrodynamics simulations of binary neutron star mergers with varying realistic EOS and mass-ratios.  With these simulations we established that previous results using different codes were qualitatively correct, even though our neutrino schemes (amongst other things) were not identical. Talk is under way to use these simulations for new studies. Firstly, the subset of simulations which collapsed to a black hole were not evolved past this point and it would be interesting to see what emissions arose afterwards and to study the properties and evolution of the resulting accretion disks, which have been shown to produce a significant amount of ejecta when MHD is introduced (\cite{fernandez2019long}). Secondly, we initially ran each simulation with tracer particles that followed the fluid flow, but stopped this because it was slowing down the code too much to get a report out on time for this thesis. Tracer particles would allow us to better study the properties of the ejecta, by not only tracing out the fluid worldlines of the ejecta, but also by using the tracer data to seed light-curve simulations and nuclear reaction networks. All of this was done for neutron star - black-hole binaries in (\cite{fernandez2016dynamics}) by some of our collaborators. In terms of other source parameters, We also neglected in our study the eccentricity of the binary and the spins of the neutron stars, which can produce significantly more ejecta. In the future, the addition of magneto-hydrodynamics, eccentricity, spins and more realistic neutrino transport schemes will definitely be on the agenda.

In conclusion, while this thesis has helped push us closer to the goal of simulating and understanding binary neutron star mergers, a great deal of work still needs to be done. The next few decades should be a very interesting period for both numerical relativity and gravitational wave astrophysics, so stay tuned.

%% \section{Future Work and Directions}

%% Moving forward, the work in this thesis will spawn new research projects. Firstly, 
%% our discontinuous Galerkin solver will be incorporated into the numerical relativity code SpECTRE (\cite{kidder2016spectre}), which is using a task-based parallelism framwork to prepare for running on exa-scale supercomputers. Secondly, our binary neutron star simulations can be continued to long times post-merger, at which point they can be used to initialize axisymmetric simulations for late-time ejecta analysis such as in (NUCLEOSYNTHESIS paper here). Furthermore, 
