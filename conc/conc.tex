\chapter{Conclusions \& Future Work}
\label{chap:conc}

\section{Conclusions}

In this thesis, we have sought out two things. Firstly, to improve the numerical techniques used to solve the Einstein field equations. To this effect, Chapter 1 presented a completely new scheme to solve the Einstein constraints and tested it on puncture initial data, showing promising results. Secondly, we wanted to increase the knowledge-base surrounding the emissions of binary neutron star mergers by running state-of-art general relativistic neutrino-hydrodynamical simulations of binary neutron star mergers with varying realistic EOS and mass-ratios. With these simulations we established that older results with less sophisticated neutrino schemes were qualitatively correct and within error of our quantitative results.

\section{Future Work and Directions}

Moving forward, the work in this thesis will spawn new research projects. Firstly, 
our discontinuous Galerkin solver will be incorporated into the numerical relativity code SpECTRE \cite{kidder2016spectre}, which is using a task-based parallelism framwork to prepare for running on exa-scale supercomputers. Secondly, our binary neutron star simulations can be continued to long times post-merger, at which point they can be used to initialize axisymmetric simulations for late-time ejecta analysis such as in (NUCLEOSYNTHESIS paper here). Furthermore, 
