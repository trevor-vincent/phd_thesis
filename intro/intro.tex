\chapter{Introduction}
\label{chap:intro}


%https://arxiv.org/pdf/1803.07965.pdf
%https://arxiv.org/pdf/1710.05832.pdf
%https://journals.aps.org/prd/pdf/10.1103/PhysRevD.95.024029 
%Appendix D in Rezolla hydrodynamics good for Tabulated Equation of State and geometrized units
%Constraints on the neutron star equation of state from AT2017gfo using radiative transfer simulations -> good for intro paragraph
% Evaluating radiation transport errors in merger simulations using a Monte-Carlo algorithm
% https://arxiv.org/abs/1805.11581 GW170817: Measurements of neutron star radii and equation of state
% wyatts paper
% https://www.fis.unipr.it/gravity/HomePages/Thesis/PhDthesis_2009_SebastianoBernuzzi.pdf
% https://arxiv.org/pdf/1804.06308.pdf
% \subsection{Binary Neutron Star Mergers}
\cite{}
The first proposals for kilometer-scale gravitational wave interferometers were
formulated in the 1980s, and the scientific justification was based on two principal
potential sources:  the inspiral and merger of compact-object binaries, with neutron star and/or
black hole components and supernova explosions, which were later showed to be unlikely as first source detections. Since
then, the LIGO[12], Virgo[15], and GEO600[22] detectors have been funded,
built, commissioned, and operated from 2005 to 2010 as a network in their initial
stage of sensitivity – without detections at that stage. Alongside the early-2000 interferometer development,
numerical simulations of the Einstein equations were beginning to gather ground. In 2000 the first
binary neutron star coalescence was simulated, five years later the first binary black hole coalescence
was simulated and finally in 2006 the first binary neutron star - black hole simulations were performed.
A variety of numerical relativity groups started forming at this time all around the globe, the Caltech-Cornell-CITA group,
the Kyoto group, the RIT group, to name but a few. These groups started building banks of simulated waveforms
to aid in the parameter estimation studies that were expected to follow from the first detections. On September 9th, 2015, in a very unexpected event, LIGO detected the gravitational wave from two coalescing 30-$M_\odot$ black holes
in a ground-breaking 25 SNR event. Since this day, there have been four different binary black hole waveform detections. More recently however, there was the landmark discovery of GW170817. This observation
coincided with the detection of a gamma ray burst, GRB 170817A [6, 7], verifying that the source binary contained matter, which was further corroborated by a series of observations that followed across the electromagnetic spectrum, e.g. [8–12]. The measured masses of the bodies and the variety of electromagnetic observations are consistent with neutron stars (NSs). With these detections, the gravitational era of astronomy has been born. Much more is expected in the future, with KAGRA and LIGO-India, there will be a worldwide ground detector network and soon there will third generation detectors like. We will be able to see the full frequency spectrum of gravitational waves.

Even though numerical relativity has matured greatly since the major developments of the early 2000's, there is still a lot more that needs to be done. The biggest problems today in numerical relativity are the simulation of binary neutron star mergers and the simulation of supernovae, both requiring a great amount of microphysics, multi-scale grids, large sets of complex non-linear PDES, large supercomputers and modern numerical techniques. With the coming of the exascale age of supercomputing there is an increasingly realistic chance of simulating these systems with all the known microphysics. This thesis aims to tackle two distinct problems related to the first of these systems: binary neutron star mergers. 1) Further probe the parameter space of binary neutron star simulations with one the most state of the art numerical relativity codes to help understand the properties of these LIGO sources 2) Improve the computational techniques used to simulate binary neutron star mergers so that more realistic microphysics may be introduced into the simulations as computing power reaches exascale and beyond. Chapter 1 of this thesis will address the latter problem, Chapter 2 will address the former. Finally, the remaining portion of this introductory chapter will briefly describe the physics and computational techniques needed to understand the remaining two chapters.
